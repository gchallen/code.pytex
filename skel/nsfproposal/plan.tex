% 12 Dec 2011 : GWA : 1 or fewer pages.

\section{Qualifications, Deliverables, and Plan}

In this section we describe why we are qualified to undertake this project,
list project deliverables, and present a plan outlining how we will complete
the required tasks within \XXXnote{FIXME} years.

\subsection{Qualifications and Prior Support}
\label{section-qualifications}

\subsubsection{Geoffrey Challen\space} wrote a dissertation on energy
management for embedded sensor systems. Along with co-PI Hempstead, he
developed PowerTOSSIM~\cite{powertossim-sensys04}, an augmented version of
the TinyOS~\cite{tinyos-asplos00} simulator TOSSIM~\cite{tossim-sensys03}
enabling application power profiling. Challen's work on Lance, IDEA and
Peloton addressed energy consumption at the network---rather than
node---level. Lance~\cite{lance-sensys08} showed that data-intensive sensor
network applications must consider both the cost and value of information
when collecting data, and proposed a novel optimization heuristic enabling
near-optimal online performance. IDEA~\cite{idea-mobisys10} demonstrated that
a network-wide energy coordination layer could facilitate energy
optimizations impossible for a single node to perform alone.
Peloton~\cite{peloton-hotos09} proposed a distributed operating system for
coordinated resource management built on state sharing, a distributed energy
ticket abstraction, and neighborhood ticket management.

Challen's recent work focuses on smartphone usage characterization using data
collected on \PhoneLab{}~\cite{phonelab-sensemine13}. His group has designed
and deployed software to operate the testbed, and a usage characterization
experiment collecting a variety of useful data in order to begin active
public experimentation. His current and prior NSF projects include:

\begin{enumerate}

\item
\href{http://www.nsf.gov/awardsearch/showAward?AWD\_ID=1205656}{\textbf{PhoneLab:
A Programmable Participatory Smartphone Testbed}}
\textit{(CI-ADDO-NEW-1205656, \$1.3M, 06/01/2012--05/31/2015)}---Co-PI
Challen leads the \PhoneLab{} project along with co-investigators from the
University at Buffalo. \PhoneLab{} is a programmable smartphone testbed
providing the power, scale, and realism required to evaluate mobile systems
research. Consisting of 288~smartphones, \PhoneLab{} opens for public
experimentation in October, 2013.

\item
\href{http://www.nsf.gov/awardsearch/showAward?AWD\_ID=1059586}{\textbf{Travel
Support for SenSys 2010}}
\textit{(CNS-NeTS, \$15,000,
10/01/2010--09/30/2011}. Co-PI Challen distributed NSF funding supporting
student travel to SenSys'10.

\end{enumerate}

% 12 Dec 2011 : GWA : 1 page.

\subsection{Deliverables}
\label{subsec-methods}

\subsection{Plan}

\subsubsection{Year 1:\space}

\subsubsection{Year 2:\space}
